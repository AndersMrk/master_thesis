% Colors (edit palette here)
\definecolor{PipeOuter}{HTML}{000000}
\definecolor{PipeInner}{HTML}{006083}
\definecolor{AxisColor}{HTML}{000000}
\definecolor{FlowColor}{HTML}{FFFFFF}
\definecolor{LeakColor}{HTML}{006083}
\definecolor{BgColor}{HTML}{FFFFFF} % set to page/background color

\begin{tikzpicture}[
  font=\small,
  >=Latex,
  axis/.style={draw=AxisColor, line width=0.8pt, line cap=butt},
  flow/.style={draw=FlowColor, line width=1.2pt},
  leak/.style={draw=LeakColor, line width=2pt, line cap=round},
  leakArrow/.style={draw=LeakColor, line width=1.2pt},
  tick/.style={draw=AxisColor, line width=0.8pt, line cap=butt},
  breakmark/.style={inner sep=0pt, text=PipeOuter, font=\normalsize},
  crack/.style={draw=BgColor, line width=0.5pt, line join=butt, fill=BgColor, fill opacity=1},
  pipe/.style={draw=PipeOuter, line width=1.5pt, line join=round, fill=PipeInner, fill opacity=1},
  pipeEndLine/.style={draw=PipeOuter, line width=2pt, line cap=round},
]

% ---- Geometry ----
\def\Lpipe{12}      % visual length representing [0, L]
\def\yPipe{0}
\def\yAxis{1.2}

% Pipe break
\def\pipeoffset{0.15}
\def\zBreakA{3.35}
\def\zBreakB{8.65}

% ---- PIPE ----
% ---- dimensions ----
\def\Ro{0.35}   % outer half-height
\def\s{0.25}    % x-slant
\def\CapExtend{0.12} % how much the cap line extends beyond the pipe (in y)

% Tunables for the S shape
\def\ScurveA{0.18}  % horizontal control offset (amplitude)
\def\ScurveB{0.28}  % vertical control bias (where the bend happens)

% Draw an "S" between two points (top -> bottom), bowing left/right.
% For a given end edge, the curve is drawn in the edge's local frame using x/y offsets.
\newcommand{\DrawScurveEdge}[2]{%
  \draw[pipeEndLine]
    #1
      .. controls
        ($#1+(\ScurveA,-\ScurveB)$) and
        ($#2+(-\ScurveA,\ScurveB)$)
      .. #2;
}

\newcommand{\ScurveDown}[2]{%
  .. controls ($#1+(\ScurveA,-\ScurveB)$) and ($#2+(-\ScurveA,\ScurveB)$) .. #2%
}
\newcommand{\ScurveUp}[2]{%
  .. controls ($#1+(-\ScurveA,\ScurveB)$) and ($#2+(\ScurveA,-\ScurveB)$) .. #2%
}

% ---- one segment: fill between outer and inner contours ----
\newcommand{\PipeSeg}[5]{%
  % #1 xa, #2 xb, #3 y, #4 leftcap, #5 rightcap
  \def\xa{#1}\def\xb{#2}\def\y{#3}%
  \pgfmathsetmacro\Ltop{\xa}\pgfmathsetmacro\Lbot{\xa}%
  \pgfmathsetmacro\Rtop{\xb}\pgfmathsetmacro\Rbot{\xb}%
  \def\flat{flat}\def\slash{slash}%

  \edef\tmp{#4}\ifx\tmp\slash
    \pgfmathsetmacro\Ltop{\xa+\s}
    \pgfmathsetmacro\Lbot{\xa}
  \fi
  \edef\tmp{#5}\ifx\tmp\slash
    \pgfmathsetmacro\Rtop{\xb}
    \pgfmathsetmacro\Rbot{\xb-\s}
  \fi

  % define corner coordinates
  \coordinate (LT) at (\Ltop,{\y+\Ro});
  \coordinate (LB) at (\Lbot,{\y-\Ro});
  \coordinate (RT) at (\Rtop,{\y+\Ro});
  \coordinate (RB) at (\Rbot,{\y-\Ro});

  % flags (no conditionals inside the path)
  \def\Lslashed{0}\def\Rslashed{0}
  \edef\tmp{#4}\ifx\tmp\slash\def\Lslashed{1}\fi
  \edef\tmp{#5}\ifx\tmp\slash\def\Rslashed{1}\fi

  % draw + fill pipe (4 cases)
  \ifnum\Lslashed=0\relax
    \ifnum\Rslashed=0\relax
      % flat ... flat
      \draw[pipe] (LT) -- (RT) -- (RB) -- (LB) -- cycle;
    \else
      % flat ... slash
      \draw[pipe] (LT) -- (RT) \ScurveDown{(RT)}{(RB)} -- (LB) -- cycle;
    \fi
  \else
    \ifnum\Rslashed=0\relax
      % slash ... flat
      \draw[pipe] (LT) -- (RT) -- (RB) -- (LB) \ScurveUp{(LB)}{(LT)} -- cycle;
    \else
      % slash ... slash
      \draw[pipe] (LT) -- (RT) \ScurveDown{(RT)}{(RB)} -- (LB) \ScurveUp{(LB)}{(LT)} -- cycle;
    \fi
  \fi

  Draw S-shapes
  \edef\tmp{#4}\ifx\tmp\slash
    \DrawScurveEdge{(\Ltop,{\y+\Ro})}{(\Lbot,{\y-\Ro})}
  \fi
  \edef\tmp{#5}\ifx\tmp\slash
    \DrawScurveEdge{(\Rtop,{\y+\Ro})}{(\Rbot,{\y-\Ro})}
  \fi
}

% leak locations
\def\zOne{1.5}      % z_1
\def\zI{5.2}        % z_i
\def\zIp{6.8}       % z_{i+1}
\def\zN{10.5}       % z_n

% constant flow arrow length (all equal)
\def\flowlen{0.8}

% ---- Pipe ----
\pgfmathsetmacro\Y{\yPipe+\pipeoffset}
\PipeSeg{0}{\zBreakA}{\Y}{flat}{slash}
\PipeSeg{\zBreakA}{\zBreakB}{\Y}{slash}{slash}
\PipeSeg{\zBreakB}{\Lpipe}{\Y}{slash}{flat}
\draw[pipe] (0,\Y+\Ro+0.1) -- (0,\Y-\Ro-0.1);
\draw[pipe] (\Lpipe,\Y+\Ro+0.1) -- (\Lpipe,\Y-\Ro-0.1);


% ---- Z-axis above pipe ----
\draw[axis,->] (0,\yAxis) -- (\Lpipe+0.6,\yAxis) node[right] {$z$-axis};

% endpoints 0 and L
\draw[tick] (0,\yAxis-0.10) -- (0,\yAxis+0.10) node[right=-7pt, yshift=8pt,text=AxisColor] {$0 \mapsto h_{0}$};
\draw[tick] (\Lpipe,\yAxis-0.10) -- (\Lpipe,\yAxis+0.10) node[right=-7pt, yshift=8pt,text=AxisColor] {$L \mapsto h_{L}$};

% optional break markers on axis
\node[breakmark] at (\zBreakA+0.5*\s,\yAxis) {//};
\node[breakmark] at (\zBreakB+0.5*\s,\yAxis) {//};

% ---- Crack marker (polygon) at x = #1, centered on y = \yPipe ----
% Usage: \CrackAt{xcoord}
\newcommand{\CrackAt}[1]{%
  \begin{scope}[shift={(#1,\yPipe+\pipeoffset-\Ro)}]
    % crack as a closed polygon (edit points to taste)
    \def\r{-0.05}
    \filldraw[crack]
      ( \r,  -0.1) --
      ( \r-0.1,  0.15) --
      ( \r,  0.2) --
      ( \r-0.1,  0.4) --
      ( \r+0.2,  0.15) --
      ( \r+0.1,  0.1) --
      ( \r+0.2,  -0.1)-- cycle;
  \end{scope}%
}

% ---- Leak: gap + X marker + downward arrow + z-label on axis ----
\newcommand{\LeakAt}[2]{%
  \CrackAt{#1}
  \def\offset{-0.1}
  % downward leak arrow from pipe
  \draw[leakArrow,->] (#1,\yPipe+\offset) -- (#1,\yPipe+\offset-\flowlen)
    node[below=-6pt,right=3pt, text=AxisColor] {$\tilde{q}_{#2}$};
  % tick + label on z-axis
  \draw[tick] (#1,\yAxis-0.10) -- (#1,\yAxis+0.10)
    node[right=-7pt, yshift=8pt, text=AxisColor] {$z_{#2} \mapsto h_{#2}$};
}

\LeakAt{\zOne}{1}
\LeakAt{\zI}{i}
\LeakAt{\zIp}{i+1}
\LeakAt{\zN}{n}

% ---- Equal-length flow arrows (placed centered in each segment) ----
\newcommand{\FlowInSegment}[3]{%
  % segment endpoints: x_a (#1) to x_b (#2), label (#3)
  \path let \p1 = (#1,\yPipe), \p2 = (#2,\yPipe) in
    coordinate (m) at ($(\p1)!.5!(\p2)$);
  \draw[flow,->]
    ($(m)+(-0.5*\flowlen,0)$) -- ($(m)+(0.5*\flowlen,0)$)
    node[midway,above=0pt, text=FlowColor] {$#3$};
}

% segments defined by leak positions (break markers are only visual)
\FlowInSegment{\zOne-2*\flowlen}{\zOne}{q_0}
\FlowInSegment{\zOne}{\zOne+2*\flowlen}{q_1}
\FlowInSegment{\zI-2*\flowlen}{\zI}{q_{i-1}}
\FlowInSegment{\zI}{\zIp}{q_i}
\FlowInSegment{\zIp}{\zIp+2*\flowlen}{q_{i+1}}
\FlowInSegment{\zN-2*\flowlen}{\zN}{q_{n-1}}
\FlowInSegment{\zN}{\zN+2*\flowlen}{q_L}

% ---- Dimension line below z-axis: |---L---| ----
\newcommand{\AxisLength}[4][]{%
  % #1 optional extra TikZ options, #2 x_start, #3 x_end, #4 label
  \def\yDim{\yAxis-0.28} % vertical offset below axis (tune)
  \def\tickSize{0.06}
  \draw[axis,#1] (#2,\yDim) -- (#3,\yDim)
    node[midway, below=-7pt, fill=white, text=AxisColor] {$#4$};
  \draw[tick,#1] (#2,\yDim-\tickSize) -- (#2,\yDim+\tickSize);
  \draw[tick,#1] (#3,\yDim-\tickSize) -- (#3,\yDim+\tickSize);
}
% ---- Length annotations below the z-axis ----
\AxisLength{0}{\zOne}{L_1}
\AxisLength{\zI}{\zIp}{L_{i+1}}
\AxisLength{\zN}{\Lpipe}{L_{n+1}}
\end{tikzpicture}
