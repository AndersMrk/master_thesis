\chapter{Derivations?}

\section{Direct Solution Method?}

Consider a pipe with $n$ leaks, where the length of the pipe, $L$, and the resistance coefficient, $\theta$, are assumed to be known. The goal is to identify the set of unknown leak parameters $\{z_i,c_i\},\;i\in[1,n]$ given a set of boundary measurements $\mathcal{M}\in\mathbb{R}^{4 \times m}$. The leak positions $z_i$ are uniquely defined by the length of the non-leaking pipe sections $L_i$ by 
\begin{equation}
    z_i=\sum_{k=1}^i L_k.
\end{equation}
Let
\begin{equation}
    \mathbf{x} := [L_1,...,L_n,c_1,...,c_n] \in \mathbb{R}^{2n}
\end{equation}
define the vector of unknown leak parameters, constrained to
\begin{equation}
    L_i>0,\quad c_i>0, \quad \sum_{i=1}^n L_i<L.
\end{equation}

A given measurement $\mu^{(j)}\in\mathcal{M}$ defines a steady-state operating point. For this operating point, the system states $h^{(j)}_i$, $q^{(j)}_i$ and $\tilde{q}^{(j)}_i$ are constant. Following the upstream equation for head, \cref{eq:upstream_head}, $n$ equations are defined as:
\begin{equation}
\label{eq:phi_h}
    \phi_h^{(j)} = \begin{bmatrix}
    h_1^{(j)}\\[0.4em]
    h_2^{(j)}\\
    \vdots\\
    h_i^{(j)}\\
    \vdots\\
    h_{n-1}^{(j)}\\[0.4em]
    h_n^{(j)}
    \end{bmatrix}
    =
    \begin{bmatrix}
    h_0^{(j)}-\theta L_1 \left(q_0^{(j)}\right)^2\\
    h_1^{(j)}-\theta L_2 \left(q_1^{(j)}\right)^2\\
    \vdots\\
    h_{i-1}^{(j)}-\theta L_{i} \left(q_{i-1}^{(j)}\right)^2\\
    \vdots\\
    h_{n-2}^{(j)}-\theta L_{n-1} \left(q_{n-2}^{(j)}\right)^2\\
    h_{n-1}^{(j)}-\theta L_{n} \left(q_{n-1}^{(j)}\right)^2
    \end{bmatrix}
\end{equation}
Similarly, $n-1$ equations are defined by the upstream equation for flow, \cref{eq:upstream_flow}, and by inserting \cref{eq:leak_flow}: 
\begin{equation}
\label{eq:phi_q}
    \phi_q^{(j)} = \begin{bmatrix}
    q_1^{(j)}\\[0.4em]
    q_2^{(j)}\\
    \vdots\\
    q_i^{(j)}\\
    \vdots\\
    q_{n-2}^{(j)}\\[0.4em]
    q_{n-1}^{(j)}
    \end{bmatrix}
    =
    \begin{bmatrix}
    q_0^{(j)}-\tilde{q}^{(j)}_1\\[0.4em]
    q_1^{(j)}-\tilde{q}^{(j)}_2\\
    \vdots\\
    q_{i-1}^{(j)}-\tilde{q}^{(j)}_{i}\\
    \vdots\\
    q_{n-3}^{(j)}-\tilde{q}^{(j)}_{n-2}\\[0.4em]
    q_{n-2}^{(j)}-\tilde{q}^{(j)}_{n-1}
    \end{bmatrix}
    =
    \begin{bmatrix}
    q_0^{(j)} - c_1 \sqrt{h_1^{(j)}}\\[0.4em]
    q_1^{(j)} - c_2 \sqrt{h_2^{(j)}}\\
    \vdots\\
    q_{i-1}^{(j)} - c_{i} \sqrt{h_{i}^{(j)}}\\
    \vdots\\
    q_{n-3}^{(j)} - c_{n-2} \sqrt{h_{n-2}^{(j)}}\\[0.4em]
    q_{n-2}^{(j)} - c_{n-1} \sqrt{h_{n-1}^{(j)}}
    \end{bmatrix}
\end{equation}
From \cref{eq:phi_h,eq:phi_q}, it can be seen that the full set of system states
\begin{equation*}
    \phi^{(j)}=\{\phi_h^{(j)},\phi_q^{(j)}\}=\{h_1^{(j)},...,h_n^{(j)},q_1^{(j)},...,q_{n-1}^{(j)}\}\in\mathbb{R}^{2n-1}
\end{equation*}
are uniquely defined by $\mathbf{x}$ and the inlet measurements $q_0^{(j)}$ and $h_0^{(j)}$. This is shown by grouping the equations into an iterative forward pass, where each iteration is defined by $f_{fp}({\phi_{i-1}};\mathbf{x}_i): \{h_{i-1},q_{i-1}\}\mapsto\{h_{i},q_{i}\}$, illustrated by \cref{fig:single_forward_pass}. Where $\phi_{i-1}=\{h_{i-1},q_{i-1}\}$ and $\mathbf{x}_i=\{c_i,L_i\}$. As an extension of the forward pass, the estimate of outlet flow can be uniquely defined by
\begin{equation}
    \label{eq:ql_estimate}
    \hat{q}_L^{(j)}:=q_{n-1}^{(j)}-c_n\sqrt{h_n^{(j)}}
\end{equation}
\cref{fig:complete_forward_pass} illustrates the chaining of each iteration into the full forward pass $\{h_{0},q_{0}\}\mapsto\{h_{n}, \hat{q}_L\}$.

\begin{figure}[h!]
    \centering
    \begin{tikzpicture}[
  >=Latex,
  block/.style={draw, rectangle, minimum height=7mm, minimum width=10mm, align=center},
  sum/.style={draw, circle, inner sep=0pt, minimum size=6mm, font=\small},
  line/.style={-Latex, thick},
]

% --- Key y-levels ---
\def\yTop{0}
\def\yBot{-3.6}

\def\sumOne{3}
\def\sumTwo{6}
\def\lineEnd{9}

\def\rowOne{-1.2}
\def\rowTwo{-2.4}

% --- Top path nodes ---
\node (hprev) at (0,\yTop) {$h_{i-1}$};
\node[sum] (sumH) at (\sumOne,\yTop) {$\sum$};
\node (hi)   at (\lineEnd,\yTop) {$h_{i}$};

% --- Bottom path nodes ---
\node (qprev) at (0,\yBot) {$q_{i-1}$};
\node[sum] (sumQ) at (\sumTwo,\yBot) {$\sum$};
\node (qi)   at (\lineEnd,\yBot) {$q_{i}$};

% --- Left vertical branch (kept ABOVE the q-line) ---
%\node[block] (theta) at (\sumOne,-1) {$\theta$};
\node[block] (Li)    at (\sumOne,\rowOne) {$\theta L_i$};
\node[block] (sq)    at (\sumOne,\rowTwo) {$({\cdot})^2$}; % above yBot=-3.0
\node (b1)   at (\sumOne,\yBot) {};

% --- Right vertical branch from hi down into sumQ ---
\node (b2)   at (\sumTwo,\yTop) {};
\node[block] (sqrt) at (\sumTwo,\rowOne) {$\sqrt{\cdot\;}$};
\node[block] (ci)   at (\sumTwo,\rowTwo) {$c_i$};

% --- Connections: top line ---
\draw[line] (hprev) -- (sumH);
\draw[line] (sumH) -- (hi);

% --- Connections: bottom line ---
\draw[line] (qprev) -- (sumQ);
\draw[line] (sumQ) -- (qi);

% --- Left branch connections into sumH (minus on the INPUT, not inside the sum) ---
\draw[line] (b1.center) -- (sq);
\draw[line] (sq) -- (Li);
\draw[line] (Li.north) -- (sumH.south);
\node[font=\small, anchor=east] at ($(sumH.south)-(0,1.5mm)$) {$-$};


% --- Right branch: hi -> sqrt -> c_i -> sumQ (minus on the INPUT, not inside the sum) ---
\draw[line] (b2.center) -- (sqrt.north);
\draw[line] (sqrt) -- (ci);
\draw[line] (ci.south) -- (sumQ.north);
\node[font=\small, anchor=west] at ($(sumQ.north)+(0,1.5mm)$) {$-$};

\end{tikzpicture}

    \caption{Caption}
    \label{fig:single_forward_pass}
\end{figure}

\begin{figure}[h!]
    \centering
    % =======================
% Fig: forward/parallel chain with boxed f_fp blocks
% Usage: \input{<this-file>.tex}
% Requires in main preamble:
%   \usepackage{tikz}
%   \usetikzlibrary{calc,positioning}
% =======================

\begin{tikzpicture}[
  >=Latex,
  line/.style={-Latex, thick},
  fblock/.style={draw, rectangle, minimum height=15mm, minimum width=18mm, align=center},
  lab/.style={font=\small},
]

% --- Geometry ---
%\def\endP{14}
\def\yH{0.3}      % y for h-row
\def\yB{-0.3}     % y for q-row
\def\yF{0.0}      % y for f-block centers
\def\bOne{1}
\def\bTwo{5.5}
\def\bThree{10}
\def\stub{1}

% --- First block: f_{fp,0} mapping (h0,q0) -> (h1,q1) ---
\node[fblock] (f0) at (\bOne,\yF) {$f_{fp}(\mathbf{x}_1)$};

\draw[line] ($(f0.west|-0,\yH) - (\stub,0)$) -- ($(f0.west|-0,\yH)$) node[midway, above] {$h_0$};
\draw[line] ($(f0.west|-0,\yB) - (\stub,0)$) -- ($(f0.west|-0,\yB)$) node[midway, below] {$q_0$};
\draw[line] ($(f0.east|-0,\yH)$) -- ($(f0.east|-0,\yH) + (\stub,0)$) node[midway, above] {$h_1$};
\draw[line] ($(f0.east|-0,\yB)$) -- ($(f0.east|-0,\yB) + (\stub,0)$) node[midway, below] {$q_1$};

% % --- Ellipsis between early indices and i-1 ---
\node[lab] (dotsA) at (3.25,\yF) {$\cdots$};

% % --- Middle block: f_{fp,i-1} mapping (h_{i-1},q_{i-1}) -> (h_i,q_i) ---
\node[fblock] (f1) at (\bTwo,\yF) {$f_{fp}(\mathbf{x}_i)$};
\draw[line] ($(f1.west|-0,\yH) - (\stub,0)$) -- ($(f1.west|-0,\yH)$) node[midway, above] {$h_{i-1}$};
\draw[line] ($(f1.west|-0,\yB) - (\stub,0)$) -- ($(f1.west|-0,\yB)$) node[midway, below] {$q_{i-1}$};
\draw[line] ($(f1.east|-0,\yH)$) -- ($(f1.east|-0,\yH) + (\stub,0)$) node[midway, above] {$h_i$};
\draw[line] ($(f1.east|-0,\yB)$) -- ($(f1.east|-0,\yB) + (\stub,0)$) node[midway, below] {$q_i$};


% % --- Ellipsis between i and n-2 ---
\node[lab] (dotsA) at (5.5+2.25,\yF) {$\cdots$};

% % --- Last block: f_{fp,n-2} mapping (h_{n-2},q_{n-2}) -> (h_{n-1},q_{n-1}) ---
\node[fblock] (f2) at (\bThree,\yF) {$f_{fp}(\mathbf{x}_{n})$};
\draw[line] ($(f2.west|-0,\yH) - (\stub,0)$) -- ($(f2.west|-0,\yH)$) node[midway, above] {$h_{n-1}$};
\draw[line] ($(f2.west|-0,\yB) - (\stub,0)$) -- ($(f2.west|-0,\yB)$) node[midway, below] {$q_{n-1}$};
\draw[line] ($(f2.east|-0,\yH)$) -- ($(f2.east|-0,\yH) + (\stub,0)$) node[midway, above] {$h_{n}$};
\draw[line] ($(f2.east|-0,\yB)$) -- ($(f2.east|-0,\yB) + (\stub,0)$) node[midway, below] {$\hat{q}_L$};


\end{tikzpicture}

    \caption{Caption}
    \label{fig:complete_forward_pass}
\end{figure}

An important observation is that $\phi_i$ is uniquely defined only by \\$\{q_0,h_0,L_1,...,L_i,c_1,...,c_i\}$, meaning that no downstream variables $\mathbf{x}_k,\,k>i$ affects the system states $\phi_i$.

\newpage
The estimate of head at the outlet is defined by
\begin{equation}
\label{eq:hl_estimate}
    \hat{h}_L^{(j)} := h_n^{(j)} - \theta L_{n+1} \left(q_L^{(j)}\right)^2
\end{equation}
where $L_{n+1}:=L-\sum_{i=1}^nL_i$
From \cref{eq:ql_estimate,eq:hl_estimate}, the estimation error is defined as
\begin{equation}
\label{eq:solve_j}
    F^{(j)}(\mathbf{x},\phi^{(j)},\mu^{(j)}):=
    \begin{bmatrix}
        \hat{h}_L^{(j)}-h_L^{(j)}\\[0.4em]
        \hat{q}_L^{(j)}- q_L^{(j)}
    \end{bmatrix}
    =
    \begin{bmatrix}
        h_n^{(j)}-h_L^{(j)} - \theta L_{n+1} \left(q_L^{(j)}\right)^2\\
        q_{n-1}^{(j)} - q_L^{(j)} - c_n\sqrt{h_n^{(j)}}
    \end{bmatrix}
\end{equation}
A solution of the system is determined by
\begin{equation*}
    F^{(j)}(\mathbf{x},\phi^{(j)},\mu^{(j)}) = \mathbf{0} \quad \forall \;j \in [1,m]
\end{equation*}

By eliminating the state variables in \cref{eq:phi_h,eq:phi_q}, each measurement $j$ yields two scalar equations by \cref{eq:solve_j} in the global unknown vector space $\mathbf{x}\in\mathbb{R}^{2n}$. To solve for all the unknowns, there must be enough measurements at different steady-state operating points. Since each operating point yields two equations, and there are $2n$ unknown variables, it is necessary that
\begin{equation*}
    m \geq n,
\end{equation*}
assuming all $m$ measurements provide independent equations, which is discussed in \cref{some chapter}.
The complete set of equations is defined by
\begin{equation}
\label{eq:main_function}
    F(\mathbf{x},\phi,\mathcal{M}) = \begin{bmatrix}
        F^{(j)}(\mathbf{x},\phi^{(1)},\mu^{(1)})\\
        \vdots\\
        F^{(j)}(\mathbf{x},\phi^{(m)},\mu^{(m)})
    \end{bmatrix}
    \in \mathbb{R}^{2m}
    = \mathbf{0}
\end{equation}


\newpage
\section{Deriving the Derivative}

To solve \cref{eq:main_function}, it is beneficial to find the Jacobian matrix
\begin{equation*}
    J(\mathbf{x}):=\frac{\partial F(\mathbf{x},\phi,\mathcal{M})}{\partial \mathbf{x}} \in \mathbb{R}^{2m\times 2n}.
\end{equation*}
To simplify the derivation, define $q_n:=\hat{q}_L$ and let
\begin{equation*}
    F_h:= h_n - h_L - \theta L_{n+1} q_L^2,\quad F_q := q_n - q_L
\end{equation*}
which are simplified terms of \cref{eq:solve_j}
The Jacobian can then be constructed by solving 
$\frac{\partial F_h}{\partial L_i}$, 
$\frac{\partial F_h}{\partial c_i}$,
$\frac{\partial F_q}{\partial L_i}$ and
$\frac{\partial F_q}{\partial c_i}$
for $i\in[1,n]$.
Given that
\begin{equation*}
    \frac{\partial}{\partial L_i} L_{n+1} = \frac{\partial}{\partial L_i} \left(L-\sum_{i=1}^n L_i\right) = -1,
\end{equation*}
the derivatives equate to
\begin{align}
    \frac{\partial F_h}{\partial L_i} &= \frac{\partial h_n}{\partial L_i} +\theta q_L^2\\
    \frac{\partial F_h}{\partial c_i} &= \frac{\partial h_n}{\partial c_i}\\
    \frac{\partial F_q}{\partial L_i} &= \frac{\partial q_{n}}{\partial L_i}\\
    \frac{\partial F_q}{\partial c_i} &= \frac{\partial q_{n}}{\partial c_i}
\end{align}
The remaining differential equations
\begin{equation*}
    \frac{\partial h_{n}}{\partial L_i},\; 
    \frac{\partial h_{n}}{\partial c_i},\; 
    \frac{\partial q_{n}}{\partial L_i},\; 
    \frac{\partial q_{n}}{\partial c_i}
\end{equation*}
can be expressed in matrix form and solved using the chain rule by
\begin{equation*}
    \frac{\partial \phi_{n}}{\partial \mathbf{x}_i} = 
    \frac{\partial \phi_{n}}{\partial \phi_{i}} \frac{\partial \phi_i}{\partial \mathbf{x}_i} =
    \frac{\partial \phi_{n}}{\partial \phi_{n-1}} \cdots
    \frac{\partial \phi_{i+1}}{\partial \phi_{i}}
    \frac{\partial \phi_{i}}{\partial \mathbf{x}_i}
\end{equation*}
Let
\begin{equation*}
    A_i:=\frac{\partial \phi_i}{\partial \phi_{i-1}} = \frac{\partial f_{fp}(\phi_{i-1}; \mathbf{x}_i )}{\partial \phi_{i-1}} 
\end{equation*}
From the theorem about the forward pass relation to other variables (which will be defined), it follows that
\begin{equation*}
    \frac{\partial \phi_k}{\partial \phi_i} = 0,\; k<i
\end{equation*}
From \cref{eq:upstream_head}, the partial derivatives for $h_i$ are computed as
\begin{align*}
    \frac{\partial h_i}{\partial h_{i-1}} &= 1 \\
    \frac{\partial h_i}{\partial q_{i-1}} &= -2 \theta L_i q_{i-1}
\end{align*}
From \cref{eq:upstream_flow}, the derivatives are
\begin{align*}
    \frac{\partial q_i}{\partial h_{i-1}} &= \frac{\partial q_i}{\partial h_{i}}\frac{\partial h_i}{\partial h_{i-1}} = -\frac{c_i}{2 \sqrt{h_{i}}} \\
    \frac{\partial q_i}{\partial q_{i-1}} &= 1 - \frac{c_i}{2 \sqrt{h_i}}\frac{\partial h_i}{\partial q_{i-1}}
    = 1+\frac{c_i \theta L_i q_{i-1}}{\sqrt{h_i}}
\end{align*}
Finally,
\begin{equation}
\label{eq:A_i}
A_i=\begin{bmatrix}
    1 & -2 \theta L_i q_{i-1} \\[0.4em]
    \displaystyle
    -\frac{c_i}{2 \sqrt{h_{i}}} & \displaystyle 1+\frac{c_i \theta L_i q_{i-1}}{\sqrt{h_i}}
\end{bmatrix}
\end{equation}
The partial derivatives in $A_i$ can be chained for the backwards pass by defining
\begin{equation*}
    G_i := \frac{\partial \phi_n}{\partial \phi_i} = A_n A_{n-1} \cdots A_{i+1}
\end{equation*}
which can be solved through backward iteration by
\begin{equation}
    G_i = G_{i+1}A_{i+1}, \quad G_n:=\mathbf{I} 
\end{equation}
Let
\begin{equation*}
    B_i:=\frac{\partial \phi_i}{\partial \mathbf{x}_i}
\end{equation*}
From the theorem about the forward pass relation to other variables (which will be defined), it follows that
\begin{equation*}
    \frac{\partial \phi_k}{\partial \mathbf{x}_i} = 0,\; k<i
\end{equation*}
The differential equations of $B_i$ can then be solved from \cref{eq:downstream_head,eq:downstream_flow},
\begin{align*}
    \frac{\partial h_i}{\partial L_i} &= -\theta q_{i-1}^2 \\
    \frac{\partial h_i}{\partial c_i} &= 0 \\
    \frac{\partial q_i}{\partial L_i} &= - \frac{c_i}{2\sqrt{h_i}} \frac{\partial h_i}{\partial L_i} = 
    \frac{c_i \theta q_{i-1}^2}{2 \sqrt{h_i}} \\
    \frac{\partial q_i}{\partial c_i} &= -\sqrt{h_i} \\
\end{align*}
Which yields
\begin{equation}
\label{eq:B_i}
B_i=\begin{bmatrix}
    -\theta q_{i-1}^2 & 0 \\[0.4em]
    \displaystyle
    \frac{c_i \theta q_{i-1}^2}{2 \sqrt{h_i}} & -\sqrt{h_i}
\end{bmatrix}
\end{equation}
To summarise, the derivatives can be derived by solving
\begin{equation}
    \label{eq:derivative_equation}
    \frac{\partial \phi_{n}}{\partial \mathbf{x}_i} =
     A_n A_{n-1} \cdots A_{i+1} B_i =
    G_i B_i
\end{equation}



\begin{algorithm}[ht]
\caption{Forward and Backwards Pass}
\label{alg:forward_backwards_pass}
\caption{hello}

\DontPrintSemicolon

\SetKwFunction{func}{ForwardBackwardPass}
\SetKwFunction{Afunc}{CalcA}
\SetKwFunction{Bfunc}{CalcB}
\SetKwFunction{Reverse}{Reverse}
\SetKwFunction{concat}{Concatinate}
\SetKwFunction{return}{Return}
\SetKwProg{Fn}{Function}{:}{}
\SetKwInput{KwNote}{\normalfont Note}
\Fn{\func{$L,c,h_0,q_0,n,\theta$}}{
    \KwNote{\textit{$L[0]$ and $c[0]$ are dummy values to make indexing consistent with equations.}}
    \;
    $h \gets [h_0], \; q \gets [q_0]$\;
    \tcp{Forward pass}
    \For{$i \gets 1$ \KwTo n}{
        $h.\mathrm{append}\!\left(h[i-1] - \theta \cdot L[i] \cdot q[i-1]^2\right)$ \tcp*[r]{\cref{eq:downstream_head}}

        $q.\mathrm{append}\!\left(q[i-1] - c[i] \cdot \sqrt{h[i]}\right)$ \tcp*[r]{\cref{eq:downstream_flow}}
    }
    \tcp{Backward pass}
    G $\gets I$,$\;$ J $\gets$ [ ] \;
    \For{$i \gets n$ \KwTo $1$}{
        B $\gets$ \Bfunc{$c[i],h[i],q[i-1],\theta$} \tcp*[r]{\cref{eq:B_i}}
        $\text{J}.\mathrm{append}\!\left( G \cdot B\right)$ \;
        \If{$i > 1$}{
            A $\gets$ \Afunc{$L[i],c[i],h[i],q[i-1],\theta$} \tcp*[r]{\cref{eq:A_i}}
            G $\gets$ G $\cdot$ A
        }
    }
    J $\gets$ \concat{\Reverse{\normalfont J}} \tcp*[r]{\normalfont J $\in \mathbb{R}^{2\times2n}$ \qquad\quad \,} \;
    \return $h$, $q$, J \; 
}
\end{algorithm}