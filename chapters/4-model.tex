\chapter{Pipe and Leak Model}
\label{chap:model}

\section{Modelling of Pipe With Leaks}

Under the assumption that water in a closed straight pipe flows one-dimensionally, the velocity distribution is uniform over the cross-section of the pipe, and the material of the pipe is linearly elastic \footnote{Linearly elastic means that stress is proportional to strain and is true for most pipe materials such as metal, concrete, and wood \cite{chaudhry2013}.}, according to \cite{chaudhry2013}, transient hydraulic behavior is described by the momentum and continuity equations:
\begin{equation}\label{eq:momentum}
\frac{1}{gA}\frac{\partial q(z,t)}{\partial t}
+ \frac{\partial h(z,t)}{\partial z}
+ \theta q^{2}(z,t) = 0
\end{equation}
\begin{equation}\label{eq:continuity}
\frac{\partial h(z,t)}{\partial t}
+ \frac{a^{2}}{gA}\frac{\partial q(z,t)}{\partial z} = 0
\end{equation}
$q(z,t)\;[m^3/s]$ is the volumetric discharge and $h(z,t)\;[m]$ is the piezometric head (form now regarded as flow and head for brevity), which is defined as
\[
h = \frac{P}{\rho g} + z_e,
\]
where $P$ is the fluid pressure, $\rho$ is the fluid density, and $z_e$ is the elevation above a chosen datum. It represents the mechanical energy per unit weight of fluid and corresponds to the height to which the fluid would rise in a piezometer connected to the pipe. Spatial and temporal variations of $h$ therefore describe pressure surges and depressions during hydraulic transients.

Furthermore, $z$ is the spatial coordinate measured along the pipe axis, $t$ denotes time, $A$ is the pipe cross-sectional area, $g$ is gravitational acceleration, and $a$ is the pressure wave speed accounting for fluid compressibility and pipe wall elasticity. 
The resistance coefficient $\theta$ accounts for quasi-steady friction losses and is given by
\begin{equation*}
\theta = \frac{f}{2 g D A^{2}},
\end{equation*}
with $f$ being the Darcy--Weisbach friction factor and $D$ the pipe diameter.

A leak in the pipe is assumed to be modeled as a valve with a constant opening. Following \cite{chaudhry2013}, the flow through a valve, denoted by $\tilde{q}$, at location $z^*$,can be expressed as
\begin{equation*}
\tilde{q}(t) = C_d A_v \sqrt{2 g \, h(z^*,t)},
\end{equation*}
where $C_d$ is the discharge coefficient and $A_v$ is the effective area of the valve opening.  

A leak located at position $z^*$ can therefore be modeled by evaluating the valve flow relation at that location, yielding
\begin{equation}
\tilde{q}(t) = c \sqrt{h(z^*,t)},
\label{eq:leak_model}
\end{equation}
where $c = C_d A_v \sqrt{2 g}$ is a constant parameter that characterizes the size and hydraulic properties of the leak.

A pipe with $n$ leaks can be modelled as $n+1$ sections of non-leaking pipes following \cref{eq:momentum,eq:continuity} with boundary conditions defined by considering the relation between head and flow both upstream and downstream of a leak at $z^*$:

\begin{align}
    \lim_{\epsilon \rightarrow0} h(z^*-\epsilon,t) &= \lim_{\epsilon \rightarrow0} h(z^*+\epsilon,t) \label{eq:head_boundary} \\
    \lim_{\epsilon \rightarrow0} q(z^*-\epsilon,t) &= \lim_{\epsilon \rightarrow0} q(z^*+\epsilon,t) + \tilde{q}(t) \label{eq:flow_boundary}
\end{align}

\newpage
\section{Notation}
\label{sec:notation}

In this section, the notation for a pipe with $n$ leaks is defined. \cref{fig:leak_notation} illustrates the variables and indexing used to describe the system.
\begin{figure}[h!]
    \centering
    % Colors (edit palette here)
\definecolor{PipeOuter}{HTML}{000000}
\definecolor{PipeInner}{HTML}{006083}
\definecolor{AxisColor}{HTML}{000000}
\definecolor{FlowColor}{HTML}{FFFFFF}
\definecolor{LeakColor}{HTML}{006083}
\definecolor{BgColor}{HTML}{FFFFFF} % set to page/background color

\begin{tikzpicture}[
  font=\small,
  >=Latex,
  axis/.style={draw=AxisColor, line width=0.8pt, line cap=butt},
  flow/.style={draw=FlowColor, line width=1.2pt},
  leak/.style={draw=LeakColor, line width=2pt, line cap=round},
  leakArrow/.style={draw=LeakColor, line width=1.2pt},
  tick/.style={draw=AxisColor, line width=0.8pt, line cap=butt},
  breakmark/.style={inner sep=0pt, text=PipeOuter, font=\normalsize},
  crack/.style={draw=BgColor, line width=0.5pt, line join=butt, fill=BgColor, fill opacity=1},
  pipe/.style={draw=PipeOuter, line width=1.5pt, line join=round, fill=PipeInner, fill opacity=1},
  pipeEndLine/.style={draw=PipeOuter, line width=2pt, line cap=round},
]

% ---- Geometry ----
\def\Lpipe{12}      % visual length representing [0, L]
\def\yPipe{0}
\def\yAxis{1.2}

% Pipe break
\def\pipeoffset{0.15}
\def\zBreakA{3.35}
\def\zBreakB{8.65}

% ---- PIPE ----
% ---- dimensions ----
\def\Ro{0.35}   % outer half-height
\def\s{0.25}    % x-slant
\def\CapExtend{0.12} % how much the cap line extends beyond the pipe (in y)

% Tunables for the S shape
\def\ScurveA{0.18}  % horizontal control offset (amplitude)
\def\ScurveB{0.28}  % vertical control bias (where the bend happens)

% Draw an "S" between two points (top -> bottom), bowing left/right.
% For a given end edge, the curve is drawn in the edge's local frame using x/y offsets.
\newcommand{\DrawScurveEdge}[2]{%
  \draw[pipeEndLine]
    #1
      .. controls
        ($#1+(\ScurveA,-\ScurveB)$) and
        ($#2+(-\ScurveA,\ScurveB)$)
      .. #2;
}

\newcommand{\ScurveDown}[2]{%
  .. controls ($#1+(\ScurveA,-\ScurveB)$) and ($#2+(-\ScurveA,\ScurveB)$) .. #2%
}
\newcommand{\ScurveUp}[2]{%
  .. controls ($#1+(-\ScurveA,\ScurveB)$) and ($#2+(\ScurveA,-\ScurveB)$) .. #2%
}

% ---- one segment: fill between outer and inner contours ----
\newcommand{\PipeSeg}[5]{%
  % #1 xa, #2 xb, #3 y, #4 leftcap, #5 rightcap
  \def\xa{#1}\def\xb{#2}\def\y{#3}%
  \pgfmathsetmacro\Ltop{\xa}\pgfmathsetmacro\Lbot{\xa}%
  \pgfmathsetmacro\Rtop{\xb}\pgfmathsetmacro\Rbot{\xb}%
  \def\flat{flat}\def\slash{slash}%

  \edef\tmp{#4}\ifx\tmp\slash
    \pgfmathsetmacro\Ltop{\xa+\s}
    \pgfmathsetmacro\Lbot{\xa}
  \fi
  \edef\tmp{#5}\ifx\tmp\slash
    \pgfmathsetmacro\Rtop{\xb}
    \pgfmathsetmacro\Rbot{\xb-\s}
  \fi

  % define corner coordinates
  \coordinate (LT) at (\Ltop,{\y+\Ro});
  \coordinate (LB) at (\Lbot,{\y-\Ro});
  \coordinate (RT) at (\Rtop,{\y+\Ro});
  \coordinate (RB) at (\Rbot,{\y-\Ro});

  % flags (no conditionals inside the path)
  \def\Lslashed{0}\def\Rslashed{0}
  \edef\tmp{#4}\ifx\tmp\slash\def\Lslashed{1}\fi
  \edef\tmp{#5}\ifx\tmp\slash\def\Rslashed{1}\fi

  % draw + fill pipe (4 cases)
  \ifnum\Lslashed=0\relax
    \ifnum\Rslashed=0\relax
      % flat ... flat
      \draw[pipe] (LT) -- (RT) -- (RB) -- (LB) -- cycle;
    \else
      % flat ... slash
      \draw[pipe] (LT) -- (RT) \ScurveDown{(RT)}{(RB)} -- (LB) -- cycle;
    \fi
  \else
    \ifnum\Rslashed=0\relax
      % slash ... flat
      \draw[pipe] (LT) -- (RT) -- (RB) -- (LB) \ScurveUp{(LB)}{(LT)} -- cycle;
    \else
      % slash ... slash
      \draw[pipe] (LT) -- (RT) \ScurveDown{(RT)}{(RB)} -- (LB) \ScurveUp{(LB)}{(LT)} -- cycle;
    \fi
  \fi

  Draw S-shapes
  \edef\tmp{#4}\ifx\tmp\slash
    \DrawScurveEdge{(\Ltop,{\y+\Ro})}{(\Lbot,{\y-\Ro})}
  \fi
  \edef\tmp{#5}\ifx\tmp\slash
    \DrawScurveEdge{(\Rtop,{\y+\Ro})}{(\Rbot,{\y-\Ro})}
  \fi
}

% leak locations
\def\zOne{1.5}      % z_1
\def\zI{5.2}        % z_i
\def\zIp{6.8}       % z_{i+1}
\def\zN{10.5}       % z_n

% constant flow arrow length (all equal)
\def\flowlen{0.8}

% ---- Pipe ----
\pgfmathsetmacro\Y{\yPipe+\pipeoffset}
\PipeSeg{0}{\zBreakA}{\Y}{flat}{slash}
\PipeSeg{\zBreakA}{\zBreakB}{\Y}{slash}{slash}
\PipeSeg{\zBreakB}{\Lpipe}{\Y}{slash}{flat}
\draw[pipe] (0,\Y+\Ro+0.1) -- (0,\Y-\Ro-0.1);
\draw[pipe] (\Lpipe,\Y+\Ro+0.1) -- (\Lpipe,\Y-\Ro-0.1);


% ---- Z-axis above pipe ----
\draw[axis,->] (0,\yAxis) -- (\Lpipe+0.6,\yAxis) node[right] {$z$-axis};

% endpoints 0 and L
\draw[tick] (0,\yAxis-0.10) -- (0,\yAxis+0.10) node[right=-7pt, yshift=8pt,text=AxisColor] {$0 \mapsto h_{0}$};
\draw[tick] (\Lpipe,\yAxis-0.10) -- (\Lpipe,\yAxis+0.10) node[right=-7pt, yshift=8pt,text=AxisColor] {$L \mapsto h_{L}$};

% optional break markers on axis
\node[breakmark] at (\zBreakA+0.5*\s,\yAxis) {//};
\node[breakmark] at (\zBreakB+0.5*\s,\yAxis) {//};

% ---- Crack marker (polygon) at x = #1, centered on y = \yPipe ----
% Usage: \CrackAt{xcoord}
\newcommand{\CrackAt}[1]{%
  \begin{scope}[shift={(#1,\yPipe+\pipeoffset-\Ro)}]
    % crack as a closed polygon (edit points to taste)
    \def\r{-0.05}
    \filldraw[crack]
      ( \r,  -0.1) --
      ( \r-0.1,  0.15) --
      ( \r,  0.2) --
      ( \r-0.1,  0.4) --
      ( \r+0.2,  0.15) --
      ( \r+0.1,  0.1) --
      ( \r+0.2,  -0.1)-- cycle;
  \end{scope}%
}

% ---- Leak: gap + X marker + downward arrow + z-label on axis ----
\newcommand{\LeakAt}[2]{%
  \CrackAt{#1}
  \def\offset{-0.1}
  % downward leak arrow from pipe
  \draw[leakArrow,->] (#1,\yPipe+\offset) -- (#1,\yPipe+\offset-\flowlen)
    node[below=-6pt,right=3pt, text=AxisColor] {$\tilde{q}_{#2}$};
  % tick + label on z-axis
  \draw[tick] (#1,\yAxis-0.10) -- (#1,\yAxis+0.10)
    node[right=-7pt, yshift=8pt, text=AxisColor] {$z_{#2} \mapsto h_{#2}$};
}

\LeakAt{\zOne}{1}
\LeakAt{\zI}{i}
\LeakAt{\zIp}{i+1}
\LeakAt{\zN}{n}

% ---- Equal-length flow arrows (placed centered in each segment) ----
\newcommand{\FlowInSegment}[3]{%
  % segment endpoints: x_a (#1) to x_b (#2), label (#3)
  \path let \p1 = (#1,\yPipe), \p2 = (#2,\yPipe) in
    coordinate (m) at ($(\p1)!.5!(\p2)$);
  \draw[flow,->]
    ($(m)+(-0.5*\flowlen,0)$) -- ($(m)+(0.5*\flowlen,0)$)
    node[midway,above=0pt, text=FlowColor] {$#3$};
}

% segments defined by leak positions (break markers are only visual)
\FlowInSegment{\zOne-2*\flowlen}{\zOne}{q_0}
\FlowInSegment{\zOne}{\zOne+2*\flowlen}{q_1}
\FlowInSegment{\zI-2*\flowlen}{\zI}{q_{i-1}}
\FlowInSegment{\zI}{\zIp}{q_i}
\FlowInSegment{\zIp}{\zIp+2*\flowlen}{q_{i+1}}
\FlowInSegment{\zN-2*\flowlen}{\zN}{q_{n-1}}
\FlowInSegment{\zN}{\zN+2*\flowlen}{q_L}

% ---- Dimension line below z-axis: |---L---| ----
\newcommand{\AxisLength}[4][]{%
  % #1 optional extra TikZ options, #2 x_start, #3 x_end, #4 label
  \def\yDim{\yAxis-0.28} % vertical offset below axis (tune)
  \def\tickSize{0.06}
  \draw[axis,#1] (#2,\yDim) -- (#3,\yDim)
    node[midway, below=-7pt, fill=white, text=AxisColor] {$#4$};
  \draw[tick,#1] (#2,\yDim-\tickSize) -- (#2,\yDim+\tickSize);
  \draw[tick,#1] (#3,\yDim-\tickSize) -- (#3,\yDim+\tickSize);
}
% ---- Length annotations below the z-axis ----
\AxisLength{0}{\zOne}{L_1}
\AxisLength{\zI}{\zIp}{L_{i+1}}
\AxisLength{\zN}{\Lpipe}{L_{n+1}}
\end{tikzpicture}

    \caption{Notation used for a pipe with $n$ leaks.}
    \label{fig:leak_notation}
\end{figure}

\renewcommand{\descriptionlabel}[1]{\normalfont #1}
\begin{description}[itemsep=2pt, topsep=0pt, parsep=0pt]
    \item[$L$] is the length of the pipe.
    
    \item[$z \in \lbrack 0,L \rbrack$] is the spatial coordinate along the pipe.
    
    \item[$z_i, \; i\in \lbrack 1,n \rbrack$] is the location of the $i$-th leak downstream along the pipe such that $z_i < z_{i+1} \;\forall i$. Defining $z_0=0$ and $z_{n+1}=L$ simplifies equations.

    \item[$\tilde{q}_i$] is the flow leaking out of the $i$-th leak.

    \item[$h_i:=h(z_i)$] is the head at leak location $z_i$. $h_0$ and $h_L$ are defined as the head at the inlet and outlet of the pipe.

    \item[$q_i:=q(z'), \; z'\in(z_i,z_{i+1})$] is the flow of the non-leaking pipe section after the $i$-th leak. $q_0$ and $q_L$ are defined as the inlet and outlet flow of the pipe.

    \item[$L_i:=z_i-z_{i-1}$] is the length of the non-leaking pipe section between $z_{i-1}$ and $z_i$. Defining $L_{n+1}:=L-z_n$ yields $\sum_{k=1}^{n+1} L_k = L$.

    \item[$\mu:=\{h_0,h_L,q_0,q_L\}$] is the set containing the head and flow at the inlet and outlet of the pipe, which can be measured.

    \item[$\mu^{(j)}:=\{h_0^{(j)},h_L^{(j)},q_0^{(j)},q_L^{(j)}\},\;j\in(1,m)$] is the $j$-th measurement and $m$ is the total number of measurements.

    \item[$\mathcal{M}:=\bigcup_{j=1}^{m} \mu^{(j)}$] is the combined set of all $m$ measurements.
\end{description}

\newpage
\section{Derivation of Steady-State Equations}

In a steady-state, time derivatives are zero:
\begin{equation}\label{eq:zero_derivatives}
    \frac{\partial q(z,t)}{\partial t}=0, \quad \frac{\partial h(z,t)}{\partial t}=0
\end{equation}
and the head and flow are no longer time varying.

Combining \cref{eq:momentum,eq:continuity} with \cref{eq:zero_derivatives} gives the steady state equations for a non-leaking pipe segment:
\begin{equation}\label{eq:momentum_ss}
\frac{d h(z)}{d z}
= -\theta q^{2}(z),
\end{equation}
\begin{equation}\label{eq:continuity_ss}
\frac{d q(z)}{d z} = 0.
\end{equation}
Consider a non-leaking pipe section between the leaks $z_i$ and $z_{i+1}$. It follows from \cref{eq:continuity_ss} that the flow $q_i$ is constant along the entire section. Utilizing the fundamental theorem of calculus to solve \cref{eq:momentum},  the change of head along the non-leaking section is given by
\begin{align}
    \int_{z_i}^{z_{i+1}} \frac{d h(z)}{d z} dz &= \int_{z_i}^{z_{i+1}} -\theta q_i^{2} dz  \notag \\
    h(z_{i+1}) - h(z_i) &= -\theta q_i^{2} (z_{i+1}-z_i) \notag \\ 
    h_i &= h_{i+1} + \theta L_{i+1} q_i^{2} \label{eq:head_change}.
\end{align}
From \cref{eq:leak_model,eq:head_boundary}, it follow that the flow from a leak $\tilde{q}_i$ is given by
\begin{equation*}
    \tilde{q}_i = c_i\sqrt{h_i},
\end{equation*}
where $c_i$ is the leak constant for this specific leak. The change of flow in the pipe can be determined from \cref{eq:flow_boundary} and is
\begin{equation} \label{eq:flow_change}
    q_{i} = q_{i+1}+\tilde{q}_{i+1}.
\end{equation}
Using \cref{eq:head_change,eq:flow_change}, it can easily be seen that $h_i$ and $q_i$ can be expressed both by the upstream and downstream leak. This is shown in the summary below.

\section{Summary}

\noindent\fbox{
\begin{minipage}{\textwidth}
Given a pipe operating in steady state, and under the assumptions previously stated, the following equations and \cref{fig:leak_notation_summary} describe the system around a leak at location $z_i$:
\begin{enumerate}
    \item \textbf{Piezometric Head:}  
    \begin{align}
        h_i &= h_{i+1} + \theta L_{i+1} q_i^2 \label{eq:downstream_head} \\
        h_i &= h_{i-1} - \theta L_{i} q_{i-1}^2 \label{eq:upstream_head}
    \end{align}

    \item \textbf{Leak Flow:}  
    \begin{equation}
        \tilde{q}_i = c_i\sqrt{h_i} \label{eq:leak_flow}
    \end{equation}

    \item \textbf{Flow in Non-Leaking Segment:}  
    \begin{align}
        q_{i} &= q_{i+1}+\tilde{q}_{i+1} \label{eq:downstream_flow} \\
        q_{i} &= q_{i-1}-\tilde{q}_{i} \label{eq:upstream_flow}
    \end{align}
\end{enumerate}

\medskip
\centering
\input{figures/tikz_small_pipe}
\captionsetup{hypcap=false}
\captionof{figure}{Notation used for an arbitrary section of a pipe with three leaks.}
\label{fig:leak_notation_summary}
\end{minipage}
}

\section{Extension to Non-Straight Pipes}
